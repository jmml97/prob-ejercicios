% archivo:    prob.e3.tex
% asignatura: Probabilidad
% autores:      José María Martín Luque, José Luis Ruiz Benito, Ricardo Ruiz Fernández de Alba
\documentclass[
  a4paper,
  spanish,
  12pt,
]{scrartcl}

\linespread{1.1}


%-------------------------------------------------------------------------------
%	PAQUETES
%-------------------------------------------------------------------------------

% Idioma

\usepackage[es-noindentfirst, es-tabla]{babel}

% Citas de texto en línea/bloque

\usepackage[autostyle]{csquotes}

% Matemáticas

\usepackage{amsmath, amsthm, amssymb}
\usepackage{mathtools}
\usepackage{commath}
\usepackage{xfrac}

% Fuentes personalizadas para utilizar con XeLaTeX o LuaLaTeX

\usepackage[no-math]{fontspec}
\setmainfont{Libertinus Serif}
\setsansfont{Libertinus Sans}
\setmonofont{Libertinus Mono}

\usepackage[math-style=TeX]{unicode-math}
\setmathfont{Libertinus Math}


% Configuración de microtype

\defaultfontfeatures{Ligatures=TeX,Numbers=Lining}
\usepackage[activate={true,nocompatibility},final,tracking=true,factor=1100,stretch=10,shrink=10]{microtype}
\SetTracking{encoding={*}, shape=sc}{0}

% Enlaces y colores

\usepackage{hyperref}
\usepackage{xcolor}
\hypersetup{
  colorlinks=true,
  citecolor=,
  linkcolor=,
  urlcolor=blue,
}

% Otros elementos de página

\usepackage{enumitem}
%\setlist[itemize]{leftmargin=*}
%\setlist[enumerate]{leftmargin=*}

\usepackage[labelfont=sc]{caption}

\usepackage{booktabs}
\renewcommand\arraystretch{1.5}

% Tikz

\usepackage{tikz}
\usetikzlibrary{babel}
\usepackage{float}

% Números con círculos
\newcommand*\circled[1]{\tikz[baseline=(char.base)]{
            \node[shape=circle,draw,inner sep=2pt] (char) {#1};}}

% Código

\usepackage{listings}
\lstset{
	basicstyle=\ttfamily,%
	breaklines=true,%
	captionpos=b,                    % sets the caption-position to bottom
  tabsize=2,	                   % sets default tabsize to 2 spaces
  frame=lines,
  numbers=left,
  stepnumber=1,
  aboveskip=12pt,
  showstringspaces=false,
  keywordstyle=\bfseries,
  commentstyle=\itshape,
  columns=flexible,
}
%\renewcommand{\lstlistingname}{Listado}

% ENTORNOS

\newtheoremstyle{ejercicio-style}  % Nombre del estilo
{2\topsep}                                  % Espacio por encima
{1.5\topsep}                                  % Espacio por debajo
{\itshape}                                  % Fuente del cuerpo
{0pt}                                  % Identación
{\scshape}                      % Fuente para la cabecera
{.}                                 % Puntuación tras la cabecera
{5pt plus 1pt minus 1pt}                              % Espacio tras la cabecera
{{\thmname{#1}\thmnumber{ #2}}\thmnote{ (#3)}}  % Especificación de la cabecera

\newtheoremstyle{remark-style}
{-\topsep}                                  % Espacio por encima
{2\topsep}                                  % Espacio por debajo
{}                                  % Fuente del cuerpo
{0pt}                                  % Identación
{\itshape}
{.}
{5pt plus 1pt minus 1pt}                              % Espacio tras la cabecera
{}

% Ejercicios y solución
\theoremstyle{ejercicio-style}
\newtheorem{ejer}{Ejercicio}

\theoremstyle{remark-style}
\newtheorem*{sol}{Solución}


% Márgenes
\usepackage[bottom=3.125cm, top=2.5cm, left=3.5cm, right=3.5cm, marginparwidth=70pt]{geometry}

\usepackage{hyphenat}

%-------------------------------------------------------------------------------
%	CONTENIDO
%-------------------------------------------------------------------------------

\begin{document}

\begin{flushright}
  José María Martín Luque
  
  José Luis Ruiz Benito

  Ricardo Ruiz Fernández de Alba\vspace{.5em}

  \textit{Probabilidad}

  D.\,G. en Ing. Informática y Matemáticas

  \textsc{Universidad de Granada}\vspace{.5em}

  \today\vspace{.5em}
\end{flushright}

\begin{flushleft}
  \scshape\Large Entrega 3. Ejercicio de la relación 6.
\end{flushleft}

\setcounter{ejer}{4}

\begin{ejer}
  Para predecir el resultado de una elección a la que se presentan dos candidatos, \(A\) y \(B\), se seleccionan aleatoria e independientemente \(n\) personas del censo y se les pregunta su preferencia.
  Si la probabilidad de seleccionar un votante de \(A\) es \(p\), calcular el tamaño mínimo de muestra que hay que elegir para que la frecuencia relativa de votantes de \(A\) en la muestra difiera de \(p\) menos de \(0.001\) con probabilidad mayor o igual que \(0.99\).
\end{ejer}

\begin{sol}

  Vamos a llamar \(\bar{p}\) a la frecuencia relativa de votantes de \(A\) en la muestra, pues \(p\) es la probabilidad de seleccionar a un votante de \(A\).
  Tenemos que encontrar un valor \(n\) tal que
  \[
    P[|\bar{p}-p| \leq 0.001] \geq 0.99.
  \]

  El resultado de cada persona encuestada lo modelamos como una variable aleatoria que sigue una distribución de Bernoulli \(X_i\), donde el valor 1 se da si dicha persona votará a \(A\), y 0 en caso contrario.
  Como sabemos, cada \(X_i\) tiene valor esperado \(\operatorname{E}[X_i] = p\) y varianza \(\operatorname{Var}(X_i) = p(1-p)\).
  Son además, independientes todos los \(X_i\).
  Consideramos entonces su suma, \(S_n = X_1 + X_2 + \dots + X_n\).
  Sabemos que \(\operatorname{E}(S_n) = np\) y que \(\operatorname{Var}(S_n) = np(1-p)\).

  Por el teorema central del límite, podemos definir la variable \(Z\) de la forma
  \[
    Z = \frac{S_n - np}{\sqrt{np(1-p)}} \sim N(0,1).
  \]
  Además, por definición, \(\bar{p} = S_n/n\). Sustituyendo en lo anterior,
  \[
    P\left[\left|\frac{S_n}{n} - p\right| \leq 0.001\right] = 0.99\,,
  \]
  luego
  \[
    P\left[\left|S_n - np\right| \leq 0.001n\right] = 0.99\,.
  \]
  Tipificando, tenemos que
  \[
    P\left[
      \left|\frac{S_n - np}{\sqrt{np(1-p)}}\right| \leq \frac{0.001n}{\sqrt{np(1-p)}}
    \right] = 0.99.
  \]
  Por tanto, \(x = 0.001n/\sqrt{np(1-p)}\) es el número tal que entre su negativo y él hay un área igual a \(0.99\) en la distribución normal \((0,1)\), luego
  \[
    F(x) - (1-F(x)) = 0.99
  \]
  y por tanto \(F(x) = 0.995\). Buscando en la tabla encontramos que \(x = 2.58\). Ahora debemos despejar el valor de \(n\). Tenemos que
  \[
    2.58 = \frac{0.001n}{\sqrt{np(1-p)}},
  \]
  de donde obtenemos que
  \[
    n = \frac{2.58^2 p(1-p)}{0.001^2}.
  \]
  Vemos por tanto que el número de encuestados de la muestra dependerá de la probabilidad \(p\) de elegir un votante de \(A\).
\end{sol}

\end{document}
