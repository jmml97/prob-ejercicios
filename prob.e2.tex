% archivo:    prob.e1.tex
% asignatura: Probabilidad
% autores:      José María Martín Luque, José Luis Ruiz Benito, Ricardo Ruiz Fernández de Alba
\documentclass[
  a4paper,
  spanish,
  12pt,
]{scrartcl}

\linespread{1.1}


%-------------------------------------------------------------------------------
%	PAQUETES
%-------------------------------------------------------------------------------

% Idioma

\usepackage[es-noindentfirst, es-tabla]{babel}

% Citas de texto en línea/bloque

\usepackage[autostyle]{csquotes}

% Matemáticas

\usepackage{amsmath, amsthm, amssymb}
\usepackage{mathtools}
\usepackage{commath}
\usepackage{xfrac}

% Fuentes personalizadas para utilizar con XeLaTeX o LuaLaTeX

\usepackage[no-math]{fontspec}
\setmainfont{Libertinus Serif}
\setsansfont{Libertinus Sans}
\setmonofont{Libertinus Mono}

\usepackage[math-style=TeX]{unicode-math}
\setmathfont{Libertinus Math}


% Configuración de microtype

\defaultfontfeatures{Ligatures=TeX,Numbers=Lining}
\usepackage[activate={true,nocompatibility},final,tracking=true,factor=1100,stretch=10,shrink=10]{microtype}
\SetTracking{encoding={*}, shape=sc}{0}

% Enlaces y colores

\usepackage{hyperref}
\usepackage{xcolor}
\hypersetup{
  colorlinks=true,
  citecolor=,
  linkcolor=,
  urlcolor=blue,
}

% Otros elementos de página

\usepackage{enumitem}
%\setlist[itemize]{leftmargin=*}
%\setlist[enumerate]{leftmargin=*}

\usepackage[labelfont=sc]{caption}

\usepackage{booktabs}
\renewcommand\arraystretch{1.5}

% Tikz

\usepackage{tikz}
\usetikzlibrary{babel}
\usepackage{float}

% Código

\usepackage{listings}
\lstset{
	basicstyle=\ttfamily,%
	breaklines=true,%
	captionpos=b,                    % sets the caption-position to bottom
  tabsize=2,	                   % sets default tabsize to 2 spaces
  frame=lines,
  numbers=left,
  stepnumber=1,
  aboveskip=12pt,
  showstringspaces=false,
  keywordstyle=\bfseries,
  commentstyle=\itshape,
  columns=flexible,
}
%\renewcommand{\lstlistingname}{Listado}

% ENTORNOS

\newtheoremstyle{ejercicio-style}  % Nombre del estilo
{2\topsep}                                  % Espacio por encima
{1.5\topsep}                                  % Espacio por debajo
{\itshape}                                  % Fuente del cuerpo
{0pt}                                  % Identación
{\scshape}                      % Fuente para la cabecera
{.}                                 % Puntuación tras la cabecera
{5pt plus 1pt minus 1pt}                              % Espacio tras la cabecera
{{\thmname{#1}\thmnumber{ #2}}\thmnote{ (#3)}}  % Especificación de la cabecera

\newtheoremstyle{remark-style}
{-\topsep}                                  % Espacio por encima
{2\topsep}                                  % Espacio por debajo
{}                                  % Fuente del cuerpo
{0pt}                                  % Identación
{\itshape}
{.}
{5pt plus 1pt minus 1pt}                              % Espacio tras la cabecera
{}

% Ejercicios y solución
\theoremstyle{ejercicio-style}
\newtheorem{ejer}{Ejercicio}

\theoremstyle{remark-style}
\newtheorem*{sol}{Solución}


% Márgenes
\usepackage[bottom=3.125cm, top=2.5cm, left=3.5cm, right=3.5cm, marginparwidth=70pt]{geometry}

\usepackage{hyphenat}

%-------------------------------------------------------------------------------
%	CONTENIDO
%-------------------------------------------------------------------------------

\begin{document}

\begin{flushright}
  José María Martín Luque
  
  José Luis Ruiz Benito

  Ricardo Ruiz Fernández de Alba
  \vspace{.5em}

  \textit{Probabilidad}

  D.\,G. en Ing. Informática y Matemáticas

  \textsc{Universidad de Granada}\vspace{.5em}

  \today\vspace{.5em}
\end{flushright}

\begin{flushleft}
  \scshape\Large Entrega 2. Ejercicio de la relación 4.
\end{flushleft}

\setcounter{ejer}{13}

\begin{ejer}
  Sea \((X, Y)\) un vector aleatorio distribuido uniformemente en el paralelogramo de vértices \((0,0)\); \((2,0)\); \((3,1)\); \((1,1)\). Calcular el error cuadrático medio asociado a la predicción de \(X\) a partir de la variable \(Y\) y a la predicción de \(Y\) a partir de la variable aleatoria \(X\). Determinar la predicción más fiable a la vista de los resultados obtenidos.
\end{ejer}

\begin{sol}
  Lo primero que necesitamos es hallar la función de densidad del vector aleatorio. 
  Teniendo en cuenta la región en la que está definido, que podemos ver en la Figura \ref{fig:region}, y sabiendo que se distribuye uniformemente, integramos la función de densidad:
  \[
    \int_0^1\int_y^{y+2} k\dif x\dif y = \int_0^1 \big[kx\big]_{y}^{y+2} \dif y= \int_0^1 2k \dif y = 2k.
  \]
  Como debe integrar \(1\), concluimos que \(k = 1/2\).
  \begin{figure}[h]
    \centering
    \includegraphics{ej4-dibujo}
    \caption{Región del plano en la que está distribuido el vector aleatorio \((X,Y)\).}
    \label{fig:region}
  \end{figure}
  
  Ahora debemos calcular las distribuciones marginales de \(X\) e \(Y\). La de \(X\) viene dada por la función
  \[
    f_X(x) = \int_{-\infty}^{\infty} f(x,y) \dif y = \begin{dcases}
      \int_0^x 1/2 \dif y = x/2 & \text{para } 0 < x < 1,\\
      \int_0^1 1/2 \dif y = 1/2  & \text{para } 1 < x < 2,\\
      \int_{x+2}^1 1/2 \dif y = (3 - x)/2 & \text{para } 2 < x < 3.
    \end{dcases}
  \]
  La distribución marginal de \(Y\) viene dada por la función
  \[
    f_Y(y) = \int_{-\infty}^{\infty} f(x,y) \dif x = \int_y^{y+2} 1/2 \dif x = 1 \quad\text{para } 0 < y < 1.
  \]

  Una vez calculadas las distribuciones marginales, calculamos las distribuciones condicionadas. La de \(Y\) condicionado a \(X\) está determinada por la función
  \begin{align*}
    f_{y\mid x}(y) = \begin{dcases}
      \hfil\frac{1/2}{x/2} = 1/x & \text{para } 0 < x < 1,\; 0 < y < x;\\
      \hfil\frac{1/2}{1/2} = 1  & \text{para } 1 < x < 2,\; 0 < y < 1;\\
      \frac{1/2}{(3-x)/2} = 1/(3-x) & \text{para } 2 < x < 3,\; x-2 < y < 1.
    \end{dcases}
  \end{align*}
  Por otro lado, la de \(X\) condicionado a \(Y\) está determinada por la función
  \[
    f_{x \mid y}(x) = \frac{1/2}{1} = 1/2 \quad\text{para } 0 < y < 1,\; y < x < y+2.
  \]

\end{sol}

\end{document}
