% archivo:    prob.e1.tex
% asignatura: Probabilidad
% autores:      José María Martín Luque, José Luis Ruiz Benito, Ricardo Ruiz Fernández de Alba
\documentclass[
  a4paper,
  spanish,
  12pt,
]{scrartcl}

\linespread{1.1}


%-------------------------------------------------------------------------------
%	PAQUETES
%-------------------------------------------------------------------------------

% Idioma

\usepackage[es-noindentfirst, es-tabla]{babel}

% Citas de texto en línea/bloque

\usepackage[autostyle]{csquotes}

% Matemáticas

\usepackage{amsmath, amsthm, amssymb}
\usepackage{mathtools}
\usepackage{commath}
\usepackage{xfrac}

% Fuentes personalizadas para utilizar con XeLaTeX o LuaLaTeX

\usepackage[no-math]{fontspec}
\setmainfont{Libertinus Serif}
\setsansfont{Libertinus Sans}
\setmonofont{Libertinus Mono}

\usepackage[math-style=TeX]{unicode-math}
\setmathfont{Libertinus Math}


% Configuración de microtype

\defaultfontfeatures{Ligatures=TeX,Numbers=Lining}
\usepackage[activate={true,nocompatibility},final,tracking=true,factor=1100,stretch=10,shrink=10]{microtype}
\SetTracking{encoding={*}, shape=sc}{0}

% Enlaces y colores

\usepackage{hyperref}
\usepackage{xcolor}
\hypersetup{
  colorlinks=true,
  citecolor=,
  linkcolor=,
  urlcolor=blue,
}

% Otros elementos de página

\usepackage{enumitem}
%\setlist[itemize]{leftmargin=*}
%\setlist[enumerate]{leftmargin=*}

\usepackage[labelfont=sc]{caption}

\usepackage{booktabs}
\renewcommand\arraystretch{1.5}

% Tikz

\usepackage{tikz}
\usetikzlibrary{babel}
\usepackage{float}

% Números con círculos
\newcommand*\circled[1]{\tikz[baseline=(char.base)]{
            \node[shape=circle,draw,inner sep=2pt] (char) {#1};}}

% Código

\usepackage{listings}
\lstset{
	basicstyle=\ttfamily,%
	breaklines=true,%
	captionpos=b,                    % sets the caption-position to bottom
  tabsize=2,	                   % sets default tabsize to 2 spaces
  frame=lines,
  numbers=left,
  stepnumber=1,
  aboveskip=12pt,
  showstringspaces=false,
  keywordstyle=\bfseries,
  commentstyle=\itshape,
  columns=flexible,
}
%\renewcommand{\lstlistingname}{Listado}

% ENTORNOS

\newtheoremstyle{ejercicio-style}  % Nombre del estilo
{2\topsep}                                  % Espacio por encima
{1.5\topsep}                                  % Espacio por debajo
{\itshape}                                  % Fuente del cuerpo
{0pt}                                  % Identación
{\scshape}                      % Fuente para la cabecera
{.}                                 % Puntuación tras la cabecera
{5pt plus 1pt minus 1pt}                              % Espacio tras la cabecera
{{\thmname{#1}\thmnumber{ #2}}\thmnote{ (#3)}}  % Especificación de la cabecera

\newtheoremstyle{remark-style}
{-\topsep}                                  % Espacio por encima
{2\topsep}                                  % Espacio por debajo
{}                                  % Fuente del cuerpo
{0pt}                                  % Identación
{\itshape}
{.}
{5pt plus 1pt minus 1pt}                              % Espacio tras la cabecera
{}

% Ejercicios y solución
\theoremstyle{ejercicio-style}
\newtheorem{ejer}{Ejercicio}

\theoremstyle{remark-style}
\newtheorem*{sol}{Solución}


% Márgenes
\usepackage[bottom=3.125cm, top=2.5cm, left=3.5cm, right=3.5cm, marginparwidth=70pt]{geometry}

\usepackage{hyphenat}

%-------------------------------------------------------------------------------
%	CONTENIDO
%-------------------------------------------------------------------------------

\begin{document}

\begin{flushright}
  José María Martín Luque
  
  José Luis Ruiz Benito

  Ricardo Ruiz Fernández de Alba\vspace{.5em}

  \textit{Probabilidad}

  D.\,G. en Ing. Informática y Matemáticas

  \textsc{Universidad de Granada}\vspace{.5em}

  \today\vspace{.5em}
\end{flushright}

\begin{flushleft}
  \scshape\Large Entrega 2. Ejercicio de la relación 4.
\end{flushleft}

\setcounter{ejer}{13}

\begin{ejer}
  Sea \((X, Y)\) un vector aleatorio distribuido uniformemente en el paralelogramo de vértices \((0,0)\); \((2,0)\); \((3,1)\); \((1,1)\). Calcular el error cuadrático medio asociado a la predicción de \(X\) a partir de la variable \(Y\) y a la predicción de \(Y\) a partir de la variable aleatoria \(X\). Determinar la predicción más fiable a la vista de los resultados obtenidos.
\end{ejer}

\begin{sol}
  Comenzamos recordando que si estamos estimando el valor de \(X\) a partir de una función \(\varphi\) de \(Y\), podemos calcular el error cuadrático medio de dicha estimación utilizando la expresión
  \[
    \operatorname{ECM}(\varphi(Y)) = \operatorname{E}[X^2] - \operatorname{E}[\operatorname{E}[X\mid Y]^2].
  \]
  En consecuencia, necesitamos hallar:
  \begin{enumerate}[
    label=\protect\circled{\arabic*},
    wide,
    labelwidth=!, 
    labelindent=0pt
  ]
    \item La función de densidad conjunta del vector aleatorio.
    \item Las distribuciones marginales de \(X\) e \(Y\).
    \item Las distribuciones condicionadas \(X\mid Y\) e \(Y \mid X\).
    \item Las esperanzas \(\operatorname{E}[X^2]\) y \(\operatorname{E}[Y^2]\).
    \item Las esperanzas condicionadas \(\operatorname{E}[X \mid Y]\) y \(\operatorname{E}[Y \mid X]\).
    \item Las esperanzas \(\operatorname{E}[\operatorname{E}[X \mid Y]^2]\) y \(\operatorname{E}[\operatorname{E}[Y \mid X]^2]\).
  \end{enumerate}

  Vayamos paso por paso.

  \begin{enumerate}[
    label=\protect\circled{\arabic*},
    wide,
    labelwidth=!, 
    labelindent=0pt,
    listparindent=\parindent,
    parsep=0pt,
  ]
    \item Sabemos que el vector aleatorio se distribuye de forma uniforme, luego la función de densidad conjunta será de la forma \(f(x,y) = k\), con \(k \in \mathbb R\). Debemos hallar el valor de \(k\) teniendo en cuenta la región en la que está definido, que podemos ver en la Figura \ref{fig:region}. Para ello integramos
    \[
      \int_0^1\int_y^{y+2} k\dif x\dif y = \int_0^1 \big[kx\big]_{y}^{y+2} \dif y= \int_0^1 2k \dif y = 2k.
    \]
    Como se trata de una función de densidad dicha integral debe tener valor \(1\), luego \(k = 1/2\).
    \begin{figure}[h]
      \centering
      \includegraphics{ej4-dibujo}
      \caption{Región del plano en la que está distribuido el vector aleatorio \((X,Y)\).}
      \label{fig:region}
    \end{figure}
    \item La distribución marginal de \(X\) viene dada por la función de densidad
    \[
    f_X(x) = \int_{-\infty}^{\infty} f(x,y) \dif y = \begin{dcases}
      \int_0^x 1/2 \dif y = x/2 & \text{para } 0 < x < 1,\\
      \int_0^1 1/2 \dif y = 1/2  & \text{para } 1 < x < 2,\\
      \int_{x+2}^1 1/2 \dif y = (3 - x)/2 & \text{para } 2 < x < 3.
    \end{dcases}
    \]
    La distribución marginal de \(Y\) viene dada por la función de densidad
    \[
      f_Y(y) = \int_{-\infty}^{\infty} f(x,y) \dif x = \int_y^{y+2} 1/2 \dif x = 1 \quad\text{para } 0 < y < 1.
    \]
    \item La distribución condicionada \(X \mid Y\) está determinada por la función de densidad
    \[
      f_{x \mid y}(x) = \frac{1/2}{1} = 1/2 \quad\text{para } 0 < y < 1,\quad y < x < y+2.
    \]
    Por otro lado, la distribución condicionada \(Y \mid X\) está determinada por la función de densidad
    \begin{align*}
      f_{y\mid x}(y) = \begin{dcases}
        \hfil\frac{1/2}{x/2} = 1/x & \text{para } 0 < x < 1,\quad 0 < y < x;\\
        \hfil\frac{1/2}{1/2} = 1  & \text{para } 1 < x < 2,\quad 0 < y < 1;\\
        \frac{1/2}{(3-x)/2} = 1/(3-x) & \text{para } 2 < x < 3,\quad x-2 < y < 1.
      \end{dcases}
    \end{align*}
    \item Las esperanzas \(\operatorname{E}[X^2]\) y \(\operatorname{E}[Y^2]\) vienen dadas por
    \begin{align*}
      \operatorname{E}[X^2] &= \int_0^1 x^2\frac{x}{2}\dif x
         + \int_1^2 x^2\frac{1}{2} \dif x
         + \int_2^3 x^2\frac{3-x}{2} \dif x\\
        &= \left[\frac{x^4}{8}\right]^1_0
         + \left[\frac{x^3}{6}\right]^2_1
         + \left[\frac{x^3}{2} - \frac{x^4}{8}\right]^3_2\\
        &= \frac{1}{8} + \frac{8}{6} - \frac{1}{6} + \left(
          \left(\frac{27}{2} - \frac{81}{8}\right) - \left(\frac{8}{2} - \frac{16}{8}\right)
        \right)\\
        &= \frac{8}{3}\,,\\
      \intertext{y}
      \operatorname{E}[Y^2] &= \int_0^1 y^2\dif y
        = \left[\frac{y^3}{3}\right]^1_0
        = \frac{1}{3}\,.
    \end{align*}
    \item Las esperanzas condicionadas  \(\operatorname{E}[X \mid Y]\) y \(\operatorname{E}[Y \mid X]\) vienen dadas por
    \begin{align*}
      \operatorname{E}[X \mid Y] = \int x f_{x\mid y} \dif x = \int_{y}^{y+2} x/2 \dif x = \big[x^2\big]^{y+2}_y = 1 +y \quad \text{para } 0 < y < 1, 
    \end{align*}
    y
    \begin{align*}
      \operatorname{E}[Y \mid X] = \int y f_{y\mid x} \dif y = \begin{dcases}
        \int_0^x y\frac{1}{x} \dif y = \frac{1}{x}\left[\frac{y^2}{2}\right]^x_0 = \frac{x}{2} & \text{para } 0 < x < 1,\\
        \int_0^1 y \dif y = \left[\frac{y^2}{2}\right]_0^1 = \frac{1}{2}  & \text{para } 1 < x < 2,\\
        \int_0^x y\frac{1}{3-x} \dif y = \frac{1}{3-x}\left[\frac{y^2}{2}\right]_{x-2}^1 = \frac{x-1}{2} & \text{para } 2 < x < 3.
      \end{dcases}
    \end{align*}
    \item Finalmente, calculamos las esperanzas \(\operatorname{E}[\operatorname{E}[X \mid Y]^2]\) y \(\operatorname{E}[\operatorname{E}[Y \mid X]^2]\), que vienen dadas por
    \begin{align*}
      \operatorname{E}[\operatorname{E}[X \mid Y]^2]
       = \int_0^1 y^2(1+y) \dif y 
       = \left[\frac{y^3}{3} + \frac{y^4}{4}\right]_0^1 
       = \frac{1}{3} + \frac{1}{4} 
       = \frac{7}{12}.
    \end{align*}
    y
    \begin{align*}
      \operatorname{E}[\operatorname{E}[Y \mid X]^2]
      &= \int_0^1 x^2\frac{x}{2} \dif x 
       + \int_1^2 x^2\frac{1}{2} \dif x
       + \int_2^3 x^2\frac{x-1}{2} \dif x\\
      &= \left[\frac{x^4}{8}\right]_0^1 
       + \left[\frac{x^3}{6}\right]_1^2 
       + \left[\frac{x^4}{8} - \frac{x^3}{6}\right]_2^3\\ 
      &= \frac{1}{3} 
       + \left(\frac{8}{6} - \frac{1}{6}\right) 
       + \left(\left(\frac{81}{8} - \frac{27}{6}\right) - \left(\frac{16}{8} - \frac{8}{6}\right)\right)\\ 
      &= \frac{25}{4}.
    \end{align*}
  \end{enumerate}

  Por tanto, el error cuadrático medio obtenido al estimar \(X\) a partir de \(Y\) es 
  \[
    \operatorname{ECM}(\varphi(Y)) = \frac{8}{3} - \frac{7}{12} = \frac{25}{12},
  \]
  y el obtenido al estimar \(Y\) a partir de \(X\),
  \[
    \operatorname{ECM}(\varphi(X)) = \frac{1}{3} - \frac{25}{4} = ....
  \]
\end{sol}

\end{document}
